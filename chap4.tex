%% This is an example first chapter.  You should put chapter/appendix that you
%% write into a separate file, and add a line \include{yourfilename} to
%% main.tex, where `yourfilename.tex' is the name of the chapter/appendix file.
%% You can process specific files by typing their names in at the 
%% \files=
%% prompt when you run the file main.tex through LaTeX.
\chapter{Implementation}

\section{Cyclic Reduction}


Cyclic reduction (CR) is an algorithm invented by G. H. Golub and R. W. Hoekney [1] in the mid 1960s for solving linear systems related to the finite difference discretization of the Poisson equation over a rectangle. 
The basic idea of the CR algorithm is to eliminate half the unknowns, regroup the equations in even and odd numbered and again eliminate half the unknowns.


\subsection{The algorithm}

CR method only applies to matrices that can be represented as a (block) Toeplitz matrix; such problems often arise in implicit solutions for partial differential equations on a lattice. For example fast solvers for Poisson's equation express the problem as solving a tridiagonal matrix, discretising the solution on a regular grid. For example from 1D Poisson’s equation we have a tridiagonal matrix system while from 2D Poisson’s equation we have a block tridiagonal system.
In general large tridiagonal systems of linear equations appear in many numerical analysis applications. A tri-diagonal matrix is a matrix which only has values in the sub-, main- and super-diagonal.

For example, the tridiagonal system:

$a_i*x_{i-1} + b_i*x_i+c_i*x_{i+1} = F_{i} ,\t  i=1:1:n$